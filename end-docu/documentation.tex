\documentclass[11pt, a4paper]{article}


%---- Package -------------------------------------------------

\usepackage{paralist}
\usepackage[left=2.5cm,right=2.5cm,top=2cm,bottom=2.5cm,includeheadfoot]{geometry}
%\usepackage[a4paper, footskip = 20mm]{geometry}
\usepackage{amsmath,amsthm,amssymb}
\usepackage[ngerman]{babel}
\usepackage{fancyhdr}
\usepackage{graphicx}
\usepackage{multirow}
\usepackage[utf8]{inputenc}
\usepackage{ulem}
\usepackage{mathabx}
\usepackage{listings}
\usepackage[arrow, matrix, curve]{xy}

\begin{document}

%---- Kopf- und Fuflzeile -------------------------------------

\renewcommand{\footrulewidth}{0,2pt}


\pagestyle{fancy}
\fancyhead[L]{}
\fancyhead[C]{}
\fancyhead[R]{}
\fancyfoot[L]{}
\fancyfoot[R]{}
\fancyfoot[C]{~\\ \thepage}


%---- TITEL ---------------------------------------------------


%---- Name, Matr, Tutor und Tutorium ändern ------------------
{\normalsize
\noindent  Autor:\hfill Technische Universit"at Berlin\\
Willy Cai, 344324 \hfill Sommersemester 2018\\
Simon, ?????? \\
Gehad, ?????? \\\\
\hrule
\ \\[5cm]

}

%---- Fach und Hausarbeit ändern ------------------------------
\begin{center}
  \textbf{\LARGE The Assistent Bot} \\[0.4cm] { \Large Robocup 2018}\\[3cm] \

  \begin{minipage}{0.9\textwidth}
  This project deals with the development of assistent behavior implementation for a robot. On the basis of given libaries, we will implement different functionality for a robot named NAO. In the end NAO will understand different keywords and will respond with different activities. This paper will cover the creation of our implementation as well as its problems.
  \end{minipage}
\end{center}
%---- BEGIN YOUR TEXT! -----------------------------------------







\end{document}
